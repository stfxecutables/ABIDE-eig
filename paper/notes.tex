\documentclass[10pt]{article}
\usepackage{arxiv}

\usepackage[utf8]{inputenc} % allow utf-8 input
\usepackage[T1]{fontenc}    % use 8-bit T1 fonts
\usepackage[english]{babel} % hyphenation
\usepackage{hyperref}       % hyperlinks
\usepackage{url}            % simple URL typesetting
\usepackage{booktabs}       % professional-quality tables
\usepackage{amsfonts}       % blackboard math symbols
\usepackage{nicefrac}       % compact symbols for 1/2, etc.
\usepackage{microtype}      % microtypography
% \usepackage{lipsum}		% Can be removed after putting your text content
\usepackage{graphicx}
\usepackage{natbib}
\usepackage{doi}

\title{Deep Eigenvalue-Based Prediction of Functional Magnetic Resonance Imaging}
\date{\today}
\author{ \href{https://orcid.org/0000-0003-4733-0624}{\includegraphics[scale=0.06]{orcid.pdf}\hspace{1mm}Derek M.~Berger}\thanks{Use footnote for providing further
    information about author (webpage, alternative
    address)---\emph{not} for acknowledging funding agencies.} \\
  Department of Computer Science\\
  St. Francis Xavier University\\
  Anigonish, Nova Scotia, Canada \\
  \texttt{dberger@stfx.ca} \\
  %% examples of more authors
  \And
  \href{https://orcid.org/0000-0002-9604-3157}{\includegraphics[scale=0.06]{orcid.pdf}\hspace{1mm}Jacob ~Levman} \\
  Canada Research Chair\\
  St. Francis Xavier University\\
  Anigonish, Nova Scotia, Canada \\
  \texttt{jlevman@stfx.ca} \\
  %% \AND
  %% Coauthor \\
  %% Affiliation \\
  %% Address \\
  %% \texttt{email} \\
  %% \And
  %% Coauthor \\
  %% Affiliation \\
  %% Address \\
  %% \texttt{email} \\
  %% \And
  %% Coauthor \\
  %% Affiliation \\
  %% Address \\
  %% \texttt{email} \\
}

\begin{document}
\maketitle

%%%%%%%%%%%%%%%%%%%%%%%%%%%%%%%%%%%%%%%%%%%%%%%%%%%%%%%%%%%%%%%%%%%%%%%%%%%%%%%
\begin{abstract}
  Functional magnetic resonance imaging (fMRI) data has considerable potential
  for predicting neuropsychological and neurophysiological disorders. However, processing this data
  for use in machine learning (ML) and/or deep learning (DL) algorithms is challenging. In this
  paper we implement both novel deep learning architectures and a preprocessing step for converting
  fMRI images into spatially-rich 4D summary images with the same or greater predictive potential as
  the raw fMRI. We show that tuned models on these summary "eigen-perturbation" images are just as
  accurate as on the fMRI original images, despite a \(10-20\) times reduction in size, and achieve
  state-of-the-art accuracy of [over 70\% we hope!] on a \emph{large}, cross-site validation set.
\end{abstract}
\keywords{fMRI \and ABIDE \and deep learing \and convolutional neural network \and random matrix theory \and eigenvalues \and perturbation}
%%%%%%%%%%%%%%%%%%%%%%%%%%%%%%%%%%%%%%%%%%%%%%%%%%%%%%%%%%%%%%%%%%%%%%%%%%%%%%%

\section{Introduction}

So far the best accuracy is \citep{heinsfeldIdentificationAutismSpectrum2018}.

\section{Headings: first level}
\label{sec:headings}

See Section \ref{sec:headings}.

\subsection{Headings: second level}
\paragraph{Paragraph}

A footnote \footnote{Sample of the first footnote.}.

\bibliographystyle{unsrtnat}
\bibliography{references}

\end{document}